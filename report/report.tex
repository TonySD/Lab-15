\documentclass[a4paper, 12pt]{article}
\usepackage{geometry}
\usepackage[russian]{babel}
\usepackage[T2A]{fontenc}
\usepackage{chngcntr}
\usepackage{graphicx}
\usepackage{verbatim}

\begin{document}


\begin{titlepage}

\begin{center}
{\textsc{\textbf{Правительство Российской Федерации}}}\\
\vspace{0.5cm}
\hrule
\vspace{0.5cm}
{\textsc{Федеральное государственное автономное образовательное учреждение\\высшего образования <<Национальный исследовательский университет\\<<Высшая школа экономики>>}}\\
\vspace{1cm}
Кафедра <<Компьютерная безопасность>>
\end{center}

\vspace{\fill}
% Поменять номер лабы
\begin{center}
{\Large{\textbf{ОТЧЕТ \\ К ЛАБОРАТОРНОЙ РАБОТЕ №15}}} \\
\vspace{1em}
{\textbf{по дисциплине}} \\
\vspace{1em}
{\large{\textbf{<<Языки программирования>>}}}
\end{center}

\vspace{\fill}


\begin{flushright}
  \begin{minipage}[center]{15cm}

    \begin{minipage}[left]{5cm}
      {Работу выполнил\\студент группы СКБ-222}
    \end{minipage}
    \begin{minipage}[center]{5cm}
      \vspace{1.25cm}
      \hrulefill\\[-1cm]
      \begin{center}{подпись, дата}\end{center}
    \end{minipage}
    \begin{minipage}[right]{4cm}
      \vspace{0.4cm}
      \begin{flushright}{А.С. Вагин}\end{flushright}
    \end{minipage}
    \\
    \\
    \\
    \begin{minipage}[left]{5cm}
      {Работу проверил}
    \end{minipage}
    \begin{minipage}[center]{5cm}
      \vspace{1.25cm}
      \hrulefill\\[-1cm]
      \begin{center}{подпись, дата}\end{center}
    \end{minipage}
    \begin{minipage}[right]{4cm}
      \begin{flushright}{С.А. Булгаков}\end{flushright}
    \end{minipage}
  \end{minipage}
\end{flushright}

\vspace{\fill}

\begin{center}
Москва~2023
\end{center}
\end{titlepage}
\setcounter{page}{2}
\setcounter{secnumdepth}{5}
\setcounter{tocdepth}{5}

% Содержание
\tableofcontents
\cleardoublepage

\setcounter{section}{1}
\counterwithout{subsection}{section}
\graphicspath{ {./images/} }

% Постановка задачи
\cleardoublepage
\section*{Постановка задачи}\addcontentsline{toc}{section}{Постановка задачи}
Разработать класс реализующий пул потоков.
\\ \\
В основной функции продемонстрировать работу класса на основе задания 
лабораторной работы №14.
\cleardoublepage



\section*{Основная часть}\addcontentsline{toc}{section}{Основная часть}

\subsection{Описание изменений}
Была изменена функция \textit{ParseChecksum}. Ее основную работу теперь выполняет
функция \textit{run} класса \textit{ThreadPool}. Функционал измененной функции теперь отвечает
за инициализацию \textit{ThreadPool} и добавление туда заданий.

\subsection{Класс \textbf{ThreadPool}}
Был разработан класс \textit{ThreadPool}, отвечающий за параллельное выполнение заданий.
Его функционал ограничивается задачами лабораторной работы №14. 

\subsubsection{Добавление задач}
Функция \textit{addTask} принимает на вход структуру вида \textit{task}. В виду реализации
функция ничего не возвращает. Вывод осуществляет непосредственно сам поток в функции
\textit{run}.

\subsection{Внутреннее хранилище задач}
Задачи хранятся в массива типа \textit{query} (FIFO), что позволяет эффективно распределять
их между потоками. При любых взаимодействиях с массивом, доступ блокируется при помощи
\textit{std::mutex}, во избежание ошибок.

\subsection{Работа потока}
Потоки работают с создания объекта \textit{ThreadPool} и до его удаления. Если
внутренний массив задач не пуст, то поток вынимает оттуда \textit{task}, после чего
начинает его обрабатывать. При получении результата, поток выводит полученные данные, 
предварительно заблокировав вывод при помощи \textbf{std::mutex}. 

\subsection{Удаление \textit{ThreadPool}}
При вызове деструктора, объект дожидается, когда внутренний массив задач окажется пустым,
после чего меняет значение переменной \textit{finished} на \textbf{true}.
В таком случае, потоки заканчивают оставшиеся задачи и завершаются, после чего 
объект и все внутренние потоки будут удалены.

% Про новые функции

\cleardoublepage


\section*{Приложение A}\addcontentsline{toc}{section}{Приложение А}
\renewcommand\thesection{\Alph{section}}
\renewcommand\thesubsection{\thesection.\arabic{subsection}}
\setcounter{subsection}{0}

\subsection{UML-диаграмма \textit{ThreadPool}}
\includegraphics{uml.png}


\cleardoublepage

\setcounter{subsection}{0}
\section*{Приложение B}\addcontentsline{toc}{section}{Приложение B}
\renewcommand\thesection{\Alph{section}}
\renewcommand\thesubsection{B.\arabic{subsection}}

\subsection{Код файла \textit{ThreadPool}}

\fontsize{9}{9}\selectfont
% Код сюда
\begin{verbatim}
#include <atomic>
#include <future>
#include <thread>
#include <mutex>
#include <vector>
#include <queue>
#include "hash.hpp"
#include "all_structures.hpp"



class ThreadPool {
    private:
        std::atomic_bool finished;
        unsigned int available_threads;
        std::vector<std::thread> threads;
        std::queue<task> task_queue;
        std::mutex queue_mutex;
        std::mutex output_mutex;

        void run() {
            while (!finished) {
                std::unique_lock<std::mutex> queue_lock(queue_mutex);
                if (!task_queue.empty()) {
                    task workingTask = std::move(task_queue.front());
                    task_queue.pop();
                    queue_lock.unlock();
                    auto bytes = getBytesFromFile(workingTask.filename);
                    std::string result = HashFactory::hash(bytes.data(), 
                    bytes.size(), workingTask.algorithm);
                    output_mutex.lock();
                    if (workingTask.expected_hash == "") {
                        std::cout << "File " << workingTask.filename << 
                        " hash (" << algorithmToText(workingTask.algorithm) << "): " << std::endl;
                        std::cout << result << std::endl << std::endl;
                    } else if (workingTask.expected_hash == result) {
                        std::cout << "File " << workingTask.filename << 
                        " is complete! (" << algorithmToText(workingTask.algorithm) << ")" << std::endl << std::endl;
                    } else {
                        std::cout << "File " << workingTask.filename << 
                        " is corrupted! (" << algorithmToText(workingTask.algorithm) << ")" << std::endl;
                        std::cout << "Expected hash: " << workingTask.expected_hash << std::endl;
                        std::cout << "Got hash: " << result << std::endl << std::endl;
                    }
                    output_mutex.unlock();
                }
            }
        }
        public:
            ThreadPool(): finished(false) {
                available_threads = std::thread::hardware_concurrency();
                threads.reserve(available_threads);
                for (unsigned int i = 0; i < available_threads; ++i) {
                    threads.emplace_back(&ThreadPool::run, this);
                }
            }
            ThreadPool(unsigned int number_of_threads): finished(false), 
            available_threads(number_of_threads) {
                threads.reserve(available_threads);
                for (unsigned int i = 0; i < available_threads; ++i) {
                    threads.emplace_back(&ThreadPool::run, this);
                }
            }
            ThreadPool(const ThreadPool &) = delete;
            ThreadPool(ThreadPool &&) = delete;
            ~ThreadPool() {
                while (!task_queue.empty()) {}
                finished = true;
                for (auto& i : threads) {
                    i.join();
                }
            }

            void addTask(task new_task) {
                queue_mutex.lock();
                task_queue.push(new_task);
                queue_mutex.unlock();
            }
};
\end{verbatim}

\subsection{Код файла \textit{main.cpp}}

\fontsize{9}{9}\selectfont
% Код сюда
\begin{verbatim}
#include <iostream>
#include <fstream>
#include <filesystem>
#include <string>
#include <future>
#include <mutex>
#include <exception>
#include "hash.hpp"
#include "filetools.cpp"
#include "thread_pool.hpp"


std::mutex output;

// Compare Hashes with expected or output
void comparingHashes(std::map<Hashes, std::vector<file>> files) {
    std::vector<Hashes> all_hashes { Hashes::MD5, Hashes::CRC32, Hashes::SHA256 };
    std::vector<std::future<std::string>> futures;
    ThreadPool thread_pool;
    std::string got_string;
    
    for (Hashes algorithm : all_hashes) {
        for (file current : files[algorithm]) {
            thread_pool.addTask(task(current, algorithm));
        }
    }
}

int main(int argc, const char** argv) {
    if (std::filesystem::exists("Checksum.ini")) {
        comparingHashes(parseChecksum("Checksum.ini"));
    }

    if (argc < 4 && !std::filesystem::exists("Checksum.ini")) std::cout << 
    "Usage: " << argv[0] << " filenames -a (crc32/md5/sha256)" << std::endl;
    else if (argc >= 4) {
        comparingHashes(parseArguments(argc, argv));
    }

    return 0;
}
\end{verbatim}

\end{document}